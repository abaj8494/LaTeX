\documentclass[12pt]{article}
%
%   Sample template for MATH3611 assignment
%
%  These are some standard packages to load. Leave them here
%  unless you have a good reason to delete them.
%
\usepackage{amsmath} \usepackage{amsfonts} \usepackage{amssymb,amsthm,array,verbatim,graphicx,tikz}
\usepackage[top=2.5cm, bottom=2.5cm, left=2.5cm, right=2.5cm]{geometry}
%
%  You can make your life easier by defining shortcuts (macros) for special
%  symbols that you use a lotm like \R for the symbol for the real line.
%
\newcommand{\N}{{\mathbb{N}}}
\newcommand{\C}{{\mathbb{C}}}
\newcommand{\D}{{\mathbb{D}}}
\newcommand{\F}{{\mathcal{F}}}
\newcommand{\R}{{\mathbb{R}}}
\newcommand{\Q}{{\mathbb{Q}}}
\newcommand{\T}{{\mathbb{T}}}
\newcommand{\Z}{{\mathbb{Z}}}
\newcommand{\ds}{\displaystyle}
\newcommand{\st}{\,:\,}
\renewcommand{\a}{{\mathbf a}}
\newcommand{\x}{{\mathbf x}}
\newcommand{\y}{{\mathbf y}}
\newcommand{\norm}[1]{\Vert #1 \Vert}
\renewcommand{\mod}[1]{\vert #1 \vert}
\newcommand\vecx{\boldsymbol{x}}
\newcommand\vecy{\boldsymbol{y}}
\newcommand{\zero}{\boldsymbol{0}}
\newcommand{\Arg}{\mathop{\mathrm{Arg}}}
\newcommand{\cl}{\mathop{\mathrm{cl}}}
\renewcommand{\Re}{\mathop{\mathrm{Re}}}


\parindent 0pt
%%%%%%%%%%%%%%%%%%%%%%%%%%%%%%%%%%%%%%%%%%%%%%%%%%%%%%%%%%%%%%%%%%%%%%%%%%%%%%%%%%%%%
%  The actual document starts here.
%%%%%%%%%%%%%%%%%%%%%%%%%%%%%%%%%%%%%%%%%%%%%%%%%%%%%%%%%%%%%%%%%%%%%%%%%%%%%%%%%%%%%

\begin{document}
\hfill John Chan (z123456)


\begin{center}
 \begin{LARGE}MATH3611/5705 Higher Analysis\\[1ex]
             Term 2, 2022\\[1ex]
              Assignment 1
 \end{LARGE}
\end{center}


\bigskip \hrule \bigskip

\textbf{Question 3.} Give the negation of
   \[ \forall q< 0\ \exists r > 7\ \forall t > 3, \ |t^2 - q^3| > r.\]

\textbf{Solution.} The negation of the statement is
  \[ \forall q > 0\ \exists r < 7\ \forall t \in (2,3), \ |t^2 - q^3| > r. \]


\bigskip

\textbf{Question 5.} Show that every Cauchy sequence is convergent.

\medskip

\textbf{Solution.}
If $b_n = \sqrt{b_{n-1}+5}$ then pigs might fly and so, by Corollary~2.3.4, $\{b_n\}$ is a Cauchy sequence. It follows that
  \[ n_k = \begin{cases}
           k,   & \text{if $k$ is prime,} \\
           k^{k^k},  & \text{otherwise.}
           \end{cases}
   \]
Thus every $x \in \R$ is quasi-obatlative. Blah, blah, etc.

By taking the subjunctive isomorphism of $\psi$, we see that
  \begin{align*}
  \int_0^x \cos(t^2) \, dt 
     &= \frac{2\pi}{\rho^2} \ln(4x) \\
     &= \sum_{j=1}^\infty B_j(x,u) J(x,u).
  \end{align*}
The conclusion then follows by a short induction proof.
\bigskip

\hrule

\bigskip

I confirm that apart from the assistance acknowledged below this assignment is all my own work
\begin{itemize}
\item	I discussed the general ideas behind solving Question 3 with Mary Smith and Chun Li.
\item	My solution to Question 5 is based on one I found on page 199 of Spivak’s Calculus.
\end{itemize}





\end{document} 
