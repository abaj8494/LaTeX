\DocumentMetadata{}
\documentclass[dvipsnames,12pt]{exam}
% compiles with
% latexmk -pdflatex=lualatex -pdf ans.tex


\usepackage[top = 2cm, bottom = 3cm, left=1.5cm, right=1.5cm]{geometry}
\usepackage{microtype}
\usepackage{fontspec}
\usepackage{amssymb}
\usepackage{titlesec}
\usepackage{multicol}
\usepackage{braket}
\usepackage{graphicx}
\graphicspath{{./img/}}
\usepackage{xcolor}
\usepackage{cancel}
\newcommand\Ccancel[2][black]{\renewcommand\CancelColor{\color{#1}}\cancel{#2}}
\usepackage{amsmath}
\usepackage{hyperref}
\usepackage{eso-pic}
\runningfooter{}{}{\thepage}
\runningheader{}{}{\scriptsize Aayush Bajaj | z5362216}

\hypersetup{
     colorlinks   = true,
     linkcolor    = RedViolet,
     citecolor    = gray
}

\titleformat{\section}{\normalfont\Large\bfseries}{{\color{RedViolet}\S}}{0.5em}{}
\titleformat{\subsection}{\normalfont\large\bfseries}{{\large \color{RedViolet}\S\S}}{0.5em}{}
\titleformat{\subsubsection}{\normalfont\bfseries}{{\color{RedViolet}\S\S\S}}{0.5em}{}

\parindent 0pt
%%% defs courtesy of Denis:
\newcommand{\N}{{\mathbb{N}}}
\newcommand{\C}{{\mathbb{C}}}
\newcommand{\D}{{\mathbb{D}}}
\newcommand{\F}{{\mathcal{F}}}
\renewcommand{\P}{{\mathcal{P}}} %careful with this, it redefines the usual P!
\newcommand{\R}{{\mathbb{R}}}
\newcommand{\Q}{{\mathbb{Q}}}
\newcommand{\T}{{\mathbb{T}}}
\newcommand{\Z}{{\mathbb{Z}}}
\newcommand{\ds}{\displaystyle}
\newcommand{\st}{\,:\,}
\renewcommand{\a}{{\mathbf a}}
\newcommand{\x}{{\mathbf x}}
\newcommand{\y}{{\mathbf y}}
\newcommand{\norm}[1]{\Vert #1 \Vert}
\renewcommand{\mod}[1]{\vert #1 \vert}
\newcommand\vecx{\boldsymbol{x}}
\newcommand\vecy{\boldsymbol{y}}
\newcommand{\zero}{\boldsymbol{0}}
\newcommand{\Arg}{\mathop{\mathrm{Arg}}}
\newcommand{\cl}{\mathop{\mathrm{cl}}}
\renewcommand{\Re}{\mathop{\mathrm{Re}}}
%%% end defs

\AddToShipoutPictureBG*{%  % Note the asterisk (*) - this is important!
  \AtPageCenter{%
    \makebox(0,0){%
      \rotatebox{45}{\textcolor{gray!30}{\fontsize{200}{120}\selectfont FINAL}}%
    }%
  }%
}

\author{Aayush Bajaj | z5362216}
\date{\today}
\title{MATH3611 | Assignment 1}

\begin{document}

\maketitle
\dotfill
\tableofcontents
\vspace{1cm}
\begin{center}
\includegraphics[width=0.2\textwidth]{logo.png}
\end{center}
\vspace{1cm}
\hrule

\newpage

\section{Question 1} \label{question1}
Prove that a non-empty set $S$ is infinite if and only if $|S| \geq |\N|$.

\subsection{Proof}
$[\implies]$ Assume $S$ is non-empty and infinite. Show that $|S| \geq |\N|$.\\

Since $S$ is infinite and non-empty we can choose an element $a_0 \in S \st S_1 = S \backslash \{a_0\}$. Note that $S_1$ is also non-empty because $S$ is not finite (its finiteness would contradict our assumption). We continue inductively with this reasoning:\\

\begin{centering}choose $a_1 \in S_1 \st S_2 = S_1 \backslash \{a_1\}$ (which is equal to $S\backslash\{a_0, a_1\}$)\\
    $$\vdots$$
\end{centering}

Generalising this \emph{Natural Number} to \emph{Set} mapping, we construct the following injective map $f: \N \to S$:

\begin{equation}
    f(n) = a_n \in S\backslash\{a_0,\ldots, a_{n-1}\}, \qquad \forall n \in \N
\end{equation}

Thus by construction, each $a_n$ is reached by a unique $n \in \N$ and $$\N \hookrightarrow S \iff |\N| \leq |S| \iff |S| \geq |\N|$$

(by definitions 1.4.2 of cardinality in the course notes). $\square$

\bigskip

$[\impliedby]$ Assume $|S| \geq |\N|$. Show that $S$ is infinite.\\

By our assumption we have $f: \N \hookrightarrow S$\footnote{definitions 1.4.2 and slide 25 ch1}.\\

Now suppose for contradiction that $S$ \emph{is finite} such that $S = \{s_0, ...,s_n\}$. Then our map must take each natural number and map it to an element of the finite set $S$; $f(i)\, \forall i \in \N$. However, this is clearly impossible since if $S$ had $n$ elements, then any injection $f: \N \to S$ would assign infinitely many $n \in \N$ to only $n$ elements of $S$, contradicting injectivity. 

Thus, no such injection can exist. Our assumption must have been false, and $S$ is indeed \emph{infinite}. $\square$

\newpage
\section{Question 2}
Prove that a set $S$ is Dedekind finite if and only if there exists some $n \in \N$ such that $S \sim \{0,1,\ldots,n\}$.

\subsection{Dedekind Infinite Definition:}
A set $S$ is Dedekind-infinite if there is a \textbf{bijection} from $S$ to a proper subset of itself. Otherwise it is Dedekind-finite.

\subsection{Proof}
$[\impliedby]$ Assume $S \sim \{0,1,\ldots,n\}, n\in\N$. Show that $S$ is Dedekind finite.\\

Suppose for contradiction that there is a bijection from $S$ to a proper subset of itself:
\begin{equation}
    f: \set{0,\ldots, n} \to \set{0,\ldots,m}, \qquad(n > m \text{ by the proper subset construction})
\end{equation}
Then $\underbrace{f(0),f(1),\ldots,f(n)}_{\text{n+1 elements}}$ which is more than the $m+1$ elements in the co-domain, and so by the Pigeonhole-principle our function cannot be injective. Thus, this voids our bijective assumption, and indeed no such mapping exists $\implies S$  is Dedekind finite. $\square$

\bigskip

$[\implies]$ Assume $S$ is Dedekind finite. Show that $S \sim \set{0, \ldots, n}$.\\

\textbf{Contrapositive:} If $S$ is infinite, then $S$ is Dedekind infinite.\\

Assume $S$ is infinite. Show that $S$ is Dedekind infinite.\\

Since $S$ is infinite we can inductively construct an injective map $f: \N \to S$ as we did in \hyperref[question1]{Question 1}:
$$f(n) = a_n \in S \backslash \set{a_0,\ldots,a_{n-1}}$$

Then we construct a bijective map $g:S \to S \backslash \set{a_0}$:
\begin{align}
    g(a_n) &= a_{n+1}\quad \forall n \in \N\\
    g(x) &= x, \quad x \in S \backslash \set{a_n: n\in \N}
\end{align}

Which is \textbf{injective} as $a_n = f(n)$ is injective $\forall n\in \N$ and our map is identical everywhere else.\\

Furthermore, this map is \textbf{surjective} since every element in the co-domain $S\backslash \set{a_0}$ is either
\begin{itemize}
    \item $a_{n+1}$, which has a image $a_n$; or,
    \item not in the sequence and thus fixed by $g$
\end{itemize}

Hence $g$ is a \textbf{bijection} from $S$ to a proper subset of itself $S \setminus \{s_0\} $, proving the \emph{contrapositive}, and therefore if \( S \) is Dedekind-finite, it must be finite. \square

\newpage
\section{Question 3}
Recall that $\P(S)$ denotes the power set of the set $S$.
\subsection{a)}\label{part3a}
Prove that the map $f: \P(\N) \to [0,1]$ given by
    \begin{equation}
        f(S) = \sum_{j\in S} 10^{-j-1},\qquad S \in \P(\N)
    \end{equation}
is well-defined and injective.

\subsubsection{Proof}
\textbf{An Equivalence}: For the remainder of this question we work with the equivalent summation from $j=[1,\infty)$:
\begin{equation}\label{notation}
    \begin{split}
    \sum_{j\in S}10^{-j-1} = \sum_{j=1}^\infty a_j 10^{-j}, \quad S \in \P(\N)\\
    \text{where } a_j = \begin{cases}
        1 & \text{if } j \in S\\
        0 & \text{otherwise}
    \end{cases}
\end{split}
\end{equation}
\textbf{Well-defined:} 
Since $j\in \N$, $j \geq 1 \implies 10^{-j} \leq 10^{-1}$. Furthermore, if we allow each $a_j$ to ``fire'' to obtain the maximum value of the summation then we get the geometric series:
\begin{align}
    \sum_{j=1}^\infty 10^{-j} &= \frac{1}{10} + \frac{1}{10^2} + \cdots \\
                              &= \frac{\frac{1}{10}}{1-\frac{1}{10}} \qquad(\text{by } S_\infty = \frac{a}{1-r}) \\
                              &= \frac{1}{9}
\end{align}

Thus, for any $S\subseteq \N$, the sum $\sum_{j\in S} 10^{-j}$ is bounded above by $\frac{1}{9}$ and $[0,\frac{1}{9}] \subseteq [0,1]$.

\bigskip
\textbf{Injectivity:} Assume $f(S) = f(T)$. Show that $S = T$.

In our \hyperref[notation]{notation}, we assume that: 
\begin{equation}
    \sum_{j=1}^\infty a_j 10^{-j} = \sum_{j=1}^\infty b_j 10^{-j}
\end{equation}

and then wish to show that $a_j = b_j \,\forall j \in \N$ which encode the sets $S$ and $T$ respectively.

\textbf{Suppose not;} i.e. that there exists at least one $j$ such that $a_j \neq b_j$.

Letting $j_0$ be the smallest such index we can have either $a_{j_0} = 1$ and $b_{j_0} = 0$ or vice-versa. But because this problem is symmetric, we shall just opt for this former choice.

Then:
\begin{align}
    \sum_{j=1}^{j_0 - 1} a_j 10^{-j} + a_{j_0} 10^{-j_0} + \sum_{j=j_0 + 1}^\infty a_j 10^{-j} &= \sum_{j=1}^{j_0 - 1} b_j 10^{-j} + b_{j_0} 10^{-j_0} + \sum_{j=j_0 + 1}^\infty b_j 10^{-j}\\
    \implies \sum_{j=1}^{j_0 - 1} a_j 10^{-j} + {\color{RedViolet}{10^{-j_0}}} + \sum_{j=j_0 + 1}^\infty a_j 10^{-j} &= \sum_{j=1}^{j_0 - 1} b_j 10^{-j} + {\color{RedViolet}{0}} + \sum_{j=j_0 + 1}^\infty b_j 10^{-j}\\
    \implies \Ccancel[RedViolet]{\sum_{j=1}^{j_0 - 1} a_j 10^{-j}} + {\color{RedViolet}{10^{-j_0}}} + \sum_{j=j_0 + 1}^\infty a_j 10^{-j} &= \Ccancel[RedViolet]{\sum_{j=1}^{j_0 - 1} b_j 10^{-j}} + \sum_{j=j_0 + 1}^\infty b_j 10^{-j}\\
\end{align}

Here we simply substituted the differing values for $a_{j_0}$, $b_{j_0}$ and canceled the equivalent sums which are the $j$'s leading up to the first differing $j_0$.

Thus we are left with:
\begin{equation}
    10^{-j_0} + \sum_{j=j_0 +1}^\infty a_j 10^{-j} = \sum_{j=j_0+1}^\infty b_j 10^{-j}
\end{equation}

But this can never hold true because at best the right-hand-side $=10^{-j_0 -1} + 10^{-j_0 -2} + \cdots = \frac{10^{-j_0}}{9}$ (by $S_\infty = \frac{a}{1-r}$), and the left-hand-side $\geq 10^{-j_0}$ ($\sum_{j=j_0+1}^\infty a_j 10^{-j} \geq 0$). Thus our assumption was false and there does not exist even one $j\in \N$ such that $a_j \neq b_j$.\\

To conclude, by producing a contradiction of the negation we have proved the original statement. $\square$

\newpage
\subsection{b)}
Use \ref{part3a}, or otherwise, to prove that $\N \not\sim [0,1]$.

\subsubsection{Proof}
By Cantor's Theorem\footnote{ch1, slide 23}: $S \not\sim \P(S)$ for any set. So $\N \not\sim \P(\N)$ and $|\P(\N)| > |\N|$, because we can inject $\N$ into $\P(\N)$ by $x \mapsto \set{x}$.\\

Hence
\begin{equation}
    |\N| < |\P(\N)| \leq |[0,1]|, \quad\text{by the injectivity of $\P(\N)$ to $[0,1]$ in \ref{part3a}}
\end{equation}

And thus $N \not\sim [0,1]$. $\square$

\newpage


\end{document}
