\underline{\uppercase{stall counting}}
\begin{itemize}
    \setlength\itemsep{0em}
    \tiny
    \item[9.5.1] After an accepted breach by the defence the stall count restarts at “Stalling one (1)”.
    \item[9.5.2] After an accepted breach by the offence the stall count restarts at maximum nine (9).
    \item[9.5.3] After a contested stall-out the stall count restarts at “Stalling eight (8)”.
    \item[9.5.4] After all other calls, including “pick”, the stall count restarts at maximum six (6). However:
        \begin{itemize}
            \item[9.5.4.1] If there is a call involving the thrower, and a separate receiving breach, and the disc is returned to the thrower, the stall count is resumed based on the outcome of the call involving the thrower.
            \item[9.5.4.2] If there is a violation called related to The Check (Section 10.), the stall count resumes at the same count that was determined prior to that violation.
        \end{itemize}
    \item[9.6] To restart a stall count “at maximum n”, where “n” is determined by 9.5.2, 9.5.4, or 20.3.6, means the
        \begin{itemize}
            \item[9.6.1] If “x” is the last agreed number fully uttered prior to the call, then the stall count resumes at “Stalling (x plus one)” or “Stalling n”, whichever of those two numbers is lower.
        \end{itemize}

            %\item 13.4. After a “stall-out” call:
            %\item 13.4.1. If the thrower still has possession of the disc, but they believe a fast count occurred in such a manner that they did not have a reasonable opportunity to call fast count before a stall-out, the play is treated as either an accepted defensive breach (9.5.1) or a contested stall-out (9.5.3).
            %\item 13.4.3. If the thrower contests a stall-out but also attempts a pass, and the pass is incomplete, then the turnover stands and play restarts with a check.
\end{itemize}
\begin{center}[b]\end{center}
