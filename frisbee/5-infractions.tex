\begin{center}\underline{\uppercase{infractions}}\end{center}

    \begin{itemize}[noitemsep]
    \tiny
    \item 15.2. A breach of the rules regarding a Marking or Travel breach is an infraction. Infractions do not stop play.
    \item 18.1. Marking Infractions (6):
        \begin{itemize}
            \item 18.1.1.1. \textbf{“Fast Count”} – the marker: starts or continues the stall count illegally OR does not start or restart the stall count with “Stalling” OR counts in less than one second intervals OR does not correctly reduce or reset the stall count when required, OR does not start the stall count from the correct number
            \item 18.1.1.2. \textbf{“Straddle”} – a line between a defensive player’s feet comes within one disc diameter of the thrower’s pivot point.
                \item 18.1.1.3. \textbf{“Disc Space”} – any part of a defensive player is less than one disc diameter away from the torso of the thrower. However, if this situation is caused solely by movement of the thrower, it is not an infraction.
                \item 18.1.1.4. \textbf{“Wrapping”} – a line between a defensive player’s hands or arms comes within one disc diameter of the thrower’s torso, or any part of the defensive player’s body is above the thrower’s pivot point. However, if this situation is caused solely by movement of the thrower, it is not an infraction.
                \item 18.1.1.5. \textbf{"Double Team"} – a defensive player other than the marker is within three (3) metres of the thrower's pivot point without also guarding another offensive player. However, merely running across this area is not a double team.
                \item 18.1.1.6. \textbf{“Vision”} – a defensive player uses any part of their body to intentionally obstruct the thrower’s vision.
                \item 18.1.2. A marking infraction may be contested by the defence, in which case play stops.
        \end{itemize}
    \item 18.2. \textbf{“Travel”} Infractions:
        \begin{itemize}
            \item 18.2.5.1. occurs if the thrower establishes a pivot point at an incorrect location, including by not reducing speed as quickly as possible after a catch, or changing direction after a catch
            \item 18.2.5.4. also occurs if the thrower fails to keep the established pivot point until releasing the disc;
            \item 18.2.2.1. if a player catches the disc while running or jumping the player may release a pass without attempting to reduce speed and without establishing a pivot point, provided that (18.2.2.1.1) they do not change direction or increase speed until they release the pass; and (18.2.2.1.2) a maximum of two additional points of contact with the ground are made after the catch and before they release the pass.
                % ask tim what happens on a contested travel infraction
        \end{itemize}
\end{itemize}
\begin{center}[5]\end{center}
