\documentclass[12pt]{article}
\usepackage[top=25mm,right=20mm,left=20mm]{geometry}


\usepackage{parskip}
\usepackage{microtype}
\usepackage{hyperref}
\usepackage{csquotes}
\usepackage{titlesec}
\usepackage{xcolor}

\titleformat{\section}{\normalfont\Large\bfseries}{{\color{purple}\S}}{0.5em}{}
\titleformat{\subsection}{\normalfont\Large\bfseries}{{\color{purple}\S\S}}{0.5em}{}
\newcommand{\aj}{A\hspace{-0.25em}\raisebox{-.35em}J}

\author{Aayush Baj\textbf{aj}}
\date{\today}
\title{Team Fanta | Captain's report}

\begin{document}
\maketitle
\dotfill
\tableofcontents
\dotfill

\newpage

\section{Raison d'\^{e}tre}
This is a simple report on the development of the players of the \textbf{Fanta} team and the evolution of their game-specific strategies over the weekend of \textsc{Bathurst 2024}.

In writing this, I actively try to disseminate the player bias from the objective Captain's report.

\section{Day 1}
Coming into Day 1, there was a degree of preparation involved\footnote{I previously released a team strats video on \href{https://youtu.be/SpE09h80q94}{YouTube}}, and Team Fanta knew their offensive (Vodka), and defensive (Gin) lines. Players had also been instructed to construct goals for the weekend, which were recited in a team meeting on the night prior to Day 1.

\subsection{Strategy}
Strategically, our offense consisted of primary handlers (\aj{}, Adam Felah, Aiden Leong), designated cutters (Siew Kin, Rohan Ranier\marginpar{\raggedright\scriptsize on paper his name looks Indian}\footnote{pronounced Ro-wan}, Emmett McGlade\footnote{an Irish immigrant, brought along by \aj{} for the selfish purposes of calling injury subs and scoring additional goal}, Rorry, Kiyomi Sano, Ching Ni Lim) and stack anchors (Kiyomi Sano, Gaia Javier, Emily Wang, David Guo). We, contrary to the initial plan played almost exclusively a \emph{vertical stack}, with no more than 5 horo\footnote{Horizontal Stack} plays.

On Defense we played our 2-3-2 force-middle, a 3-2-2 cup which evolved to a 3-3-1 cup, and later became our most powerful defense over the course of the weekend\footnote{a special thanks to Rorry's coach board which enabled this realisation}. We also played hard match from time to time - alternating between a flick and backhand force ---often times conforming to whichever sideline proved most punishing for the offense.
Within Defense, Zachary Quay was our \enquote{D-line Captain}, he did well to lead by example and called lines appropriately. The GoT (Gang of Three) - Yitao Luo, Rorry Liu and David Guo - a.k.a Rorry 1, Rorry 2 and Rorry 3\footnote{respectively}, played well as a unit, and despite some questionable turnovers, I made the decision (thanks Cherie), to keep you all as a unit.

\subsection{Results}
\begin{center}
	\begin{tabular}{lcr}
		Team & Score & Result\\
		\hline\\
		Mac Mountain Goats & 7-5 & \textsc{WIN}\\
		S L A Y & 4-11 & \textsc{LOSS}\\
		rawr xD & 5-14 & \textsc{LOSS}\\
		UOW ROAST DUCKS & 4-10 & \textsc{LOSS}\\
	\end{tabular}
\end{center}

\subsection{Retrospection}
Overall, we lost 3 games and won 1. We began to learn our team chemistry --- i.e. Em is short and so throw the frisbee's lower. Rohan is not as fast as Zach, throw the long discs shorter, etc...


\section{Day 2}
Day 2 began with many coffees for me and a loss for team \textsc{Fanta}. Despite this fact, the score-sheet does not reveal a decimation: 7-10 was the final score.

\fbox{\parbox{\textwidth}{\sffamily
	the attitude going into day 2 changed significantly: we went from strict lines and formations to more loose, and trust based offense. this was a result of fatigue and being hungover---realising that Bathurst is a for-fun tournament, and not an opportunity for \aj{} to be a dictator.}}

In mild surprise to the above, team \textsc{Fanta} performed exceptionally well, putting tallies on the first score-card and then winning the following 2 games decisively. We employed subbing groups that were working historically over the weekend: Adam Felah and I handling, the Rorries playing defense; keeping Ong and Zach together, having either Em or Gaia on. Strategically, we also employed the same tactics---me hitting Kiyo as the front of stack from a poach, Ong sooting it deep to Ching Ni, David \textsc{Exclusively} looking for Yitao despite Cherie being \textsl{mega}-free.


\subsection{Results}
\begin{center}
\begin{tabular}{lcr}
	Team & Score & Result\\
	\hline\\
	UoN Nuggies & 7-10 & \textsc{LOSS}\\
	Mac Magpies & 8-7 & \textsc{WIN}\\
	UOW CONFIT DUCKS & 10-4 & \textsc{WIN}\\
\end{tabular}
\end{center}

\newpage
\enlargethispage{.5in}
\section{MVP's}
Gan Soon Ong, Kiyomi Sano, Emily Wang, Cherie Poon and Zachary Quay were the main recipient of the \textbf{M}ost \textbf{V}aluable \textbf{P}layer awards; they were awarded appropriately and these players either played exceptionally well throughout the game, or performed exceptional feats often enough to be noticed.

In particular, Emily Wang made important and frequent cuts. Kiyomi Sano generated blocks and was a reliable midfield player on offense. Zachary is exceptionally athletic and generated many blocks. Ong is the tube outside car dealerships; his reach is fantastic, his flicks are good and his blocks were numerous. Cherie also did well to score, generate blocks and integrated herself well in the team defense.

\section{Off field}
Adam, Em and Kiyo were fantastic co-captains throughout. Emmett was good vibes to bring with. Kin drank 15 gulps from the big container - I got 8. Emmett held the big container record for the longest period of time\footnote{30 minutes}---Phillip eventually won with 17 sips.

Once again a thank-you to Rorry for his coach board, Cherie for filling up water and Ching Ni for actually coming --- I'm not sure you've realised the importance of women in frisbee yet Ching Ni, but you are worshipped here.

The \emph{interim-party\footnote{the party between Day 1 and Day 2}} was fantastic, though since it is impossible remain exclusive to the\textsc{Fanta} team players, discourse of this party is omitted; sorry Linus.

\section{Honourable mentions}
Adam Felah's chat as deep-deep in our zone's on the first day was critical. Ching Ni as break wing was useful.

When other team's put a zone defense on us, Emily was proactive in stretching the deep space well and attacking inwards toward the midfield. Cherie did well to get free on offense too.

Kin's match defense was fantastic. He majorly kept his force correct and there was at least 1 particular instance where he got an important block\footnote{against SLAY}.

Gaia's character arc was profound too. She went from dropping important discs on day 1, to getting her sh*t together on day 2 and performing completions in the endzone\footnote{these are often referred to as \emph{goals}}. I was personally so impressed by the transition and persistence that I awarded her our game disc for the weekend.

David is very coachable, well done over the weekend David.
Finally, Rohan's layouts --- impeccable form and impressive heart.
 

\end{document}
