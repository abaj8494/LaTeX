\begin{center}\underline{\uppercase{infractions}}\end{center}

\begin{itemize}
    \item 15.2. A breach of the rules regarding a Marking or Travel breach is an infraction. Infractions do not stop play.
    \item 
    \item 18.1. Marking Infractions:
    \item 
    \item 18.1.1.1. “Fast Count” – the marker:
    \item 18.1.1.1.1. starts or continues the stall count illegally,
    \item 18.1.1.1.2. does not start or restart the stall count with “Stalling”,
    \item 18.1.1.1.3. counts in less than one second intervals,
    \item 18.1.1.1.4. does not correctly reduce or reset the stall count when required, or
    \item 18.1.1.1.5. does not start the stall count from the correct number.
    \item 
    \item 18.1.1.2. “Straddle” – a line between a defensive player’s feet comes within one disc diameter of the thrower’s pivot point.
    \item 
    \item 18.1.1.3. “Disc Space” – any part of a defensive player is less than one disc diameter away from the torso of the thrower. However, if this situation is caused solely by movement of the thrower, it is not an infraction.
    \item 
    \item 18.1.1.4. “Wrapping” – a line between a defensive player’s hands or arms comes within one disc diameter of the thrower’s torso, or any part of the defensive player’s body is above the thrower’s pivot point. However, if this situation is caused solely by movement of the thrower, it is not an infraction.
    \item 
    \item 18.1.1.5. "Double Team" – a defensive player other than the marker is within three (3) metres of the thrower's pivot point without also guarding another offensive player. However, merely running across this area is not a double team.
    \item 
    \item 18.1.1.6. “Vision” – a defensive player uses any part of their body to intentionally obstruct the thrower’s vision.
    \item 
    \item 18.1.2. A marking infraction may be contested by the defence, in which case play stops.
    \item 18.1.2.1. If a pass has been completed, a contested or retracted marking infraction must be treated as a violation by the offence, and the disc must be returned to the thrower.
    \item 18.1.3. After all marking infractions listed in 18.1.1 that are not contested, the marker must resume the stall count with the number last fully uttered before the call, minus one (1).
    \item 18.1.4. The marker may not resume counting until any illegal positioning has been corrected. To do otherwise is a subsequent marking infraction.
    \item 18.1.5. Instead of calling a marking infraction, the thrower may call a marking violation and stop play if;
    \item 18.1.5.1. the stall count is not corrected,
    \item 18.1.5.2. there is no stall count,
    \item 18.1.5.3. there is an egregious marking infraction, or
    \item 18.1.5.4. there is a pattern of repeated marking infractions.
    \item 18.1.6. If a marking infraction, or a marking violation, is called and the thrower also attempts a pass before, during or after the call, the call has no consequences (unless 18.1.2.1 applies) and if the pass is incomplete, then the turnover stands.
    \item 
    \item 
    \item 
    \item 18.2. “Travel” Infractions:
    \item 18.2.1. The thrower may attempt a pass at any time as long as they are entirely in-bounds or have established an in-bounds pivot point.
    \item 18.2.1.1. However an in-bounds player who catches a pass while airborne may attempt a pass prior to contacting the ground.
    \item 18.2.2. After catching the disc, the thrower must reduce speed as quickly as possible, without changing direction, until they have established a pivot point.
    \item 18.2.2.1. However if a player catches the disc while running or jumping the player may release a pass without attempting to reduce speed and without establishing a pivot point, provided that:
    \item 18.2.2.1.1. they do not change direction or increase speed until they release the pass; and
    \item 18.2.2.1.2. a maximum of two additional points of contact with the ground are made after the catch and before they release the pass.
    \item 18.2.3. The thrower may move in any direction (pivot) only by establishing a “pivot point”, which is a specific point on the ground with which one part of their body remains in constant contact until the disc is thrown.
    \item 18.2.4. A thrower who is not standing can use any part of their body as the pivot point.
    \item 18.2.4.1. If they stand up it is not a travel, but only if a pivot point is established at the same location.
    \item 18.2.5. A travel infraction occurs if:
    \item 18.2.5.1. the thrower establishes a pivot point at an incorrect location, including by not reducing speed as quickly as possible after a catch, or changing direction after a catch;
    \item 18.2.5.2. the thrower releases a pass in breach of 18.2.2.1;
    \item 18.2.5.3. anytime the thrower must move to a specified location, the thrower does not establish a pivot point before a wind-up or throwing action begins;
    \item 18.2.5.4. the thrower fails to keep the established pivot point until releasing the disc;
    \item 18.2.5.5. a player intentionally bobbles, fumbles or delays the disc to themselves, for the sole purpose of moving in a specific direction.
    \item 18.2.6. After an accepted travel infraction is called (“travel”), play does not stop.
    \item 18.2.6.1. The thrower establishes a pivot point at the correct location, as indicated by the player who called the travel. This must occur without delay from either player involved.
    \item 18.2.6.2. Any stall count is paused, and the thrower may not throw the disc, until a pivot point is established at the correct location.
    \item 18.2.6.3. The marker does not need to say “Stalling” before resuming the stall count.
    \item 18.2.7. If, after a travel infraction but before correcting the pivot point, the thrower throws a completed pass, the defensive team may call a travel violation. Play stops and the disc is returned to the thrower. The thrower must return to the location occupied at the time of the infraction. Play must restart with a check.
    \item 18.2.8. If, after a travel infraction, the thrower throws an incomplete pass, play continues.
    \item 18.2.9. After a contested travel infraction where the thrower has not released the disc, play stops.
\end{itemize}
\begin{center}[6]\end{center}
