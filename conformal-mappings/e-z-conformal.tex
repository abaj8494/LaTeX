\documentclass[tikz, border=0.5cm]{standalone}

\usepgfmodule{nonlineartransformations} 

\makeatletter
\def\polartransformation{%
% \pgf@x will contain the radius
% \pgf@y will contain the distance
\pgfmathsincos@{\pgf@sys@tonumber\pgf@x}%
% pgfmathresultx is now the cosine of radius and
% pgfmathresulty is the sine of radius
\pgf@x=\pgfmathresultx\pgf@y%
\pgf@y=\pgfmathresulty\pgf@y%
}
\makeatother


\begin{document}

\begin{tikzpicture}

    \begin{scope}[shift={(-8,0)}]
	\draw[thick, black] (0,0) circle(2);
        \draw[->] (3,0) -- node[above] {$f(z)=e^z$} (5,0);
    \end{scope}

    % Start nonlinear transformation
    \pgftransformnonlinear{\polartransformation}

    % Draw something with this transformation in force
    \draw[thick, black] (0,0) circle(2);
    %\draw(-3.15,-2) grid[xstep=0.3,ystep=0.3] (3.15,2);

\end{tikzpicture}

\end{document}
