Last lesson you were exposed to \emph{algorithms}---specifically the ones for \textsl{addition} and \textsl{subtraction}.

Today we will look at 2 more: \textbf{multiplication} (this section) and \textbf{division} (the next section).

The structure is identical, with one operand layed ontop of the other:
\begin{figure}
    \centering
\begin{tabular}{cccccc}
     & & &3&7&8\\
    $\times$& & &2&3&7\\
    \hline\\
     & &2&6&4&6\\
     &1&1&3&4&0\\
     &7&5&6&0&0\\
    \hline\\
     &8&9&5&8&6
\end{tabular}
\end{figure}

The reason why you do not see any carry forwards is:
\begin{solutionordottedlines}[1in]
    Because there are 9 separate multiplications that occur and almost all of them will have a carry-forward of some kind.
    Omitting them here is far tidier.
\end{solutionordottedlines}

\begin{examples}
    \begin{questions}
        \Question[1] Multiply 27 by 8
        \begin{solutionorbox}[1in]
        \end{solutionorbox}
        \Question[1] \(378 \times 37\)
        \begin{solutionorbox}[1in]
        \end{solutionorbox}
        \Question[2] Multiply 389 by 46
        \begin{solutionorbox}[1.5in]
        \end{solutionorbox}
        \Question[2] Multiply 667 by 667
        \begin{solutionorbox}[1.5in]
        \end{solutionorbox}
    \end{questions}
\end{examples}

\begin{exercises}
    \begin{questions}
        \Question[12] Carry out each calculation
        \begin{parts}
            \part \(53 \times 4\)
            \begin{solutionordottedlines}[1in]
            \end{solutionordottedlines}
            \part \(19 \times 8\)
            \begin{solutionordottedlines}[1in]
            \end{solutionordottedlines}
            \part \(64 \times 7\)
            \begin{solutionordottedlines}[1in]
            \end{solutionordottedlines}
            \part \(85 \times 4\)
            \begin{solutionordottedlines}[1in]
            \end{solutionordottedlines}
            \part \(513 \times 4\)
            \begin{solutionordottedlines}[1in]
            \end{solutionordottedlines}
            \part \(819 \times 8\)
            \begin{solutionordottedlines}[1in]
            \end{solutionordottedlines}
            \part \(235 \times 7\)
            \begin{solutionordottedlines}[1in]
            \end{solutionordottedlines}
            \part \(2006 \times 7\)
            \begin{solutionordottedlines}[1in]
            \end{solutionordottedlines}
            \part \(6543 \times 7\)
            \begin{solutionordottedlines}[1in]
            \end{solutionordottedlines}
            \part \(8159 \times 4\)
            \begin{solutionordottedlines}[1in]
            \end{solutionordottedlines}
            \part \(91370 \times 9\)
            \begin{solutionordottedlines}[1in]
            \end{solutionordottedlines}
            \part \(43987 \times 6\)
            \begin{solutionordottedlines}[1in]
            \end{solutionordottedlines}
        \end{parts}
        \Question[18] Carry out each calculation, using the long multiplication method.\\
        \begin{parts}
            \part \(453 \times 24\)
            \begin{solutionordottedlines}[1in]
            \end{solutionordottedlines}
            \part \(179 \times 86\)
            \begin{solutionordottedlines}[1in]
            \end{solutionordottedlines}
            \part \(614 \times 47\)
            \begin{solutionordottedlines}[1in]
            \end{solutionordottedlines}
            \part \(895 \times 45\)
            \begin{solutionordottedlines}[1in]
            \end{solutionordottedlines}
            \part \(135 \times 27\)
            \begin{solutionordottedlines}[1in]
            \end{solutionordottedlines}
            \part \(506 \times 68\)
            \begin{solutionordottedlines}[1in]
            \end{solutionordottedlines}
            \part \(235 \times 34\)
            \begin{solutionordottedlines}[1in]
            \end{solutionordottedlines}
            \part \(5646 \times 73\)
            \begin{solutionordottedlines}[1in]
            \end{solutionordottedlines}
            \part \(91270 \times 39\)
            \begin{solutionordottedlines}[1in]
            \end{solutionordottedlines}
            \part \(762 \times 549\)
            \begin{solutionordottedlines}[1in]
            \end{solutionordottedlines}
            \part \(936 \times 564\)
            \begin{solutionordottedlines}[1in]
            \end{solutionordottedlines}
            \part \(91370 \times 109\)
            \begin{solutionordottedlines}[1in]
            \end{solutionordottedlines}
        \end{parts}
        \Question[] Calculate each of the following
        \begin{parts}
            \Part[1] Each student in a class is given 9 coloured pencils by the teacher. How many pencils does the teacher need to supply 26 students?
            \begin{solutionordottedlines}[1in]
            \end{solutionordottedlines}
            \Part[1] A packaging machine in a factory packs 893 boxes per hour. How many boxes are packed in a 12 -hour day?
            \begin{solutionordottedlines}[1in]
            \end{solutionordottedlines}
            \Part[1] A brick wall has 43 rows of 723 bricks. How many bricks are in the wall?
            \begin{solutionordottedlines}[1in]
            \end{solutionordottedlines}
            \Part[1] A publishing company packages books in boxes of 125 . How many books are there in 298 boxes?
            \begin{solutionordottedlines}[1in]
            \end{solutionordottedlines}
        \end{parts}
        \Question[2] A hall has 86 rows of 34 seats. How many seats are there in the hall?
            \begin{solutionordottedlines}[1in]
            \end{solutionordottedlines}
        \Question[2] A machine makes 257 doughnuts in an hour. How many doughnuts can it make in 13 hours?
            \begin{solutionordottedlines}[1in]
            \end{solutionordottedlines}
        \Question[6] Copy and complete the following by finding a digit for each \(\star\)
        \begin{parts}\begin{multicols}{3}
            \part
            \begin{tabular}{cccc}
                 & &$\star$&6\\
                $\times$& & &7\\
                \hline\\
                 &6&0&$\star$\\
                \hline
            \end{tabular}
            \part
            \begin{tabular}{cccc}
                 &$\star$&$\star$&9\\
                $\times$& & &3\\
                \hline\\
                 &5&0&$\star$\\
                \hline
            \end{tabular}
            \part
            \begin{tabular}{ccccc}
                & &$\star$&$\star$&$\star$\\
                $\times$& & &$\star$&3\\
                \hline\\
                 & &6&4&8\\
                  &$\star$&$\star$&$\star$&0\\
                  \hline\\
                 &4&9&6&8
            \end{tabular}
        \end{multicols}\end{parts}
        \Question[] A particular brand of lollies comes in packets of 26 . A carton contains 34 packets.
        \begin{parts} 
            \Part[1] How many lollies are there in one carton?
            \begin{solutionordottedlines}[1in]
            \end{solutionordottedlines}
            \Part[1] How many lollies are there in 30 cartons?
            \begin{solutionordottedlines}[1in]
            \end{solutionordottedlines}
        \end{parts}
        \Question[2] A trolley at an airport is loaded with 15 cases, each with the maximum allowable weight of 20 kilograms. The trolley weighs 115 kilograms. What is the maximum possible weight of the trolley and the cases?
            \begin{solutionordottedlines}[1in]
            \end{solutionordottedlines}
        \Question[2] If 25 people each own 7 pairs of shoes, and 32 people each own 8 pairs of shoes, then how many shoes do the 57 people own in total?
            \begin{solutionordottedlines}[1in]
            \end{solutionordottedlines}
        \Question[5] Calculate your age in:
        \begin{parts}
            \part months
            \begin{solutionordottedlines}[0.5in]
            \end{solutionordottedlines}
            \part weeks
            \begin{solutionordottedlines}[0.5in]
            \end{solutionordottedlines}
            \part days
            \begin{solutionordottedlines}[0.5in]
            \end{solutionordottedlines}
            \part hours
            \begin{solutionordottedlines}[0.5in]
            \end{solutionordottedlines}
            \part seconds
            \begin{solutionordottedlines}[0.5in]
            \end{solutionordottedlines}
        \end{parts}
    \end{questions}
\end{exercises}
